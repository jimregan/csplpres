\documentclass{beamer}
\usepackage[utf8x]{inputenc}
\usepackage[T1]{fontenc}
\usetheme{Copenhagen}

\title[Czech to Polish]{Shallow-transfer rule-based machine translation from Czech to Polish}

\author[Ruth, O'Regan] % (optional, for multiple authors)
{Joanna~Ruth\inst{1} \and Jimmy~O'Regan\inst{2}}
\institute
{
  \inst{1}%
  Gda\'{n}sk University of Technology \\
  {\tt joannaruth1@gmail.com}
  \and
  \inst{2}%
  Eolaistriu Technologies \\
  {\tt joregan@gmail.com}
}

\date{}

\begin{document}

\begin{frame}
\titlepage
\end{frame}

%\begin{frame}{Warning!}
%
%This presentation is only loosely related to the paper.
%
%Think of this as more like the ``Special Features'' 
%section on a DVD.
%
%\end{frame}

\section{Introduction}
\begin{frame}{Why Czech?}
\begin{itemize}
\item Budweiser
\item Pilsner
\end{itemize}
\end{frame}

\begin{frame}{Why Czech?}
\begin{itemize}
\item Budweiser $\rightarrow$ Bud\v{e}jovice.
\item Pilsner $\rightarrow$ Plze\v{n}.
\end{itemize}

Czech is the language of beer!

\end{frame}

\begin{frame}{Some Famous Czechs}
\begin{itemize}
\item Dvo\v{r}ák
\item Jan Hus
\item Kafka
\item Good King Wenceslas
\item Eva Herzigová
\item Petra Nem\v{c}ová
\item Ivana Trump
\end{itemize}
\end{frame}

\begin{frame}{Some Famous Poles}
\begin{itemize}
\item Chopin
\item Pope John Paul II
\item Copernicus
\item Marie Curie (Maria Sk\l{}odowska)
\item Ludwik Zamenhof (Esperanto)
\item Joseph Conrad
\item Roman Pola\'{n}ski
\end{itemize}
\end{frame}

\begin{frame}{Czech and Polish}
\begin{quote}
In the 10th century, Czech and Polish were still basically the same language, 
which then began to diverge from each other, but even until the 14 century, 
Czechs and Poles understood each other without problems.
\end{quote}

\begin{center}
Czech Wikipedia, \emph{Pol\v{s}tina}
\end{center}
\end{frame}

\begin{frame}{Czech and Polish}

Both Western Slavic languages:

Czech:

\begin{itemize}
\item 12 million speakers
\item Czech word in English: robot
\end{itemize}

Polish:

\begin{itemize}
\item 50 million speakers
\item Polish word in English: vodka
\end{itemize}
\end{frame}

\begin{frame}{Czech and Polish: Similarities}
\begin{itemize}
\item Medium inflected:
  \pause
  \begin{itemize}
  \item 7 cases
  \item 3 genders
  \item Animacy distinction
  \end{itemize}
\pause
\item Relatively free word order:
  \pause
  \begin{itemize}
  \item Ala ma kota
  \item Kota Ala ma
  \item Ala kota ma
  \item Ma Ala kota
  \item \ldots
  \end{itemize}
\end{itemize}
\end{frame}
\end{section}

\begin{frame}{Czech and Polish: Cases}
\begin{center}
    \begin{tabular}{ | c | l | l |}
    \hline
    Case & Czech & Polish \\ \hline

    Nominative & matka & matka \\
    Genitive & matky & matki \\
    Dative & matce & matce \\
    Acusative & matku & matk\k{e} \\
    Instrumental & matkou & matk\k{a} \\
    Locative & matce & matce \\
    Vocative & matko & matko \\
    \hline
    \end{tabular}
\end{center}
\end{frame}

\begin{frame}{Czech and Polish: NP Differences}

\begin{center}
    \begin{tabular}{ | c | l | l |}
    \hline
     & Czech & Polish \\ \hline

    Word order & adj before noun & adj before or after noun \\
    Possessive & adjectival form & genitive \\
    \hline
    \end{tabular}
\end{center}
\end{frame}

\begin{frame}{Czech and Polish: VP Differences}

\begin{center}
    \begin{tabular}{ | c | l | l |}
    \hline
     & Czech & Polish \\ \hline

    ``ought to'' & by $+$ m\'{i}t_{past} $+$ INF & powinien $+$ INF \\
    ``while {\it x}-ing'' & present transgressive (adj) & adverb {\it (-j\k{a}c)}\\
    ``having {\it x}-ed'' & past transgressive (adj) & adverb {\it (-wszy)} \\
    past tense & personal form from {\it b\'{y}t} & conjugated \\
    \hline
    \end{tabular}
\end{center}
\end{frame}

\begin{frame}{Lexical differences: A little history}

Germanisation of Bohemia began in 1620. Czech ceased to exist as a literary language.

Poland was partitioned in the 18th century. Germanisation began in the Prussian partition.

However: 

Publication allowed in the Austro-Hungarian and Russian partitions, and in France. 
Polish continued to thrive as a literary language.

\end{frame}

\begin{frame}{Lexical differences: Czech Revival}

Czech was revived in the 18th and 19th Centuries.

Jungmann's dictionary was partly based around the Bible of Kralice (16th Century),
with German words replaced by Slavic (Russian, Bulgarian) loans and neologisms.

This lead to an increase in the lexical differences between Czech and Polish.

\end{frame}

\begin{frame}{Czech vs. Polish: Viewpoints}

\begin{center}
The Czechs and Poles are neighbours, and have less-than-flattering
views of each other.
\end{center}

Polish view of Czech: Child-like

\begin{itemize}
\item More lexicalised diminutives.
\item Loss of palatalisation.
\end{itemize}

{\it\small i.e., spoken Czech sounds a little like Polish babytalk}

Czech view of Polish: Archaic

\begin{itemize}
\item Digraphs (sz, cz) instead of caron.
\item Retention of Proto-Slavic ``nasal vowels''.
\end{itemize}

{\it\small i.e., written Polish looks a little like early written Czech.}
\end{frame}

\begin{frame}{Czech View of Polish}

\begin{center}
``In Poland, a comical lisping language is spoken, dominated by different 
variants of the sound 'sh'. Polish has 17 species of them and the exact 
pronunciations are not known by the Poles themselves. \ldots
The current pronunciation of the Polish language only stabilised during 
World War II. \ldots To avoid German attacks, it could not be distinguished 
from static.''
\end{center}

\hbox{}

{\footnotesize
``V Polsku se mluví komickým šišlavým jazykem, ve kterým prevládají ruzný 
varianty hlásky "š". Polština jich má 17 druhu a jejich presnou výslovnost 
neznají ani sami Poláci. \ldots
Soucasná výslovnost polského jazyka se ustálila teprve až behem 2. svetové 
války.
\ldots
Aby nebylo pred Nemci nápadné, nesmelo být odlišitelné od statického šumu.''}

\url{http://necyklopedie.wikia.com/wiki/Polsztyna}

\end{frame}

\begin{frame}{An aside: ``l-participle''}

The Czech past form is sometimes referred to as the ``l-participle''.
Whether or not it's a participle is arguable.

\begin{itemize}
\item Not fully periphrastic: past.p3 uses no auxiliary.\footnote{The Sorbian languages do}
\item Not fully adjectival.
\item Not a modifier.
\end{itemize}
\end{frame}

\begin{frame}{Why not SMT?}

%SRSLY?

Reviewer's comment:

\begin{quote}
In section 3.4, the reader is told about the existence of a parallel corpus 
including Czech and Polish. This should be mentioned in the introduction along 
with the motivation of developing this rule-based system 
(as opposed to a statistical one).
\end{quote}

\pause

SRSLY?

\pause

For this workshop?

\pause

No prizes for spotting the obvious trolls in this section.

\end{frame}

\begin{frame}{Why not SMT?}

First and foremost:

This project was funded under Google Summer of Code: it had to produce a piece
of Open Source {\em Software}. Apertium's rules include a programmatic element;
SMT would be almost impossible to justify.

\pause

Secondly:

It's an Apertium project. 'Nuff said.

\end{frame}

\begin{frame}{Why not SMT?}
\begin{description}
  \item[Data sparseness] \hfill \\
  Compounded by relatively large amount of morphological forms.
  \item[Lack of truly parallel text] \hfill \\
  Most parallel text are mutual translations.
  \item[Lack of true corpora] \hfill \\
  JRC Acquis is not a corpus.
\end{description}

\end{frame}

\begin{frame}{Aside: JRC Acquis is not a corpus}
That might be considered a ``bold statement''.

(It's not. Ask a corpus linguist.)

\hbox{}

{\footnotesize To be clear, we're referring to the {\em Corpus}, not the DGT's Translation
Memory distribution, which, going by the Moses mailing list, is more often
used.}
\end{frame}

\begin{frame}{Aside: JRC Acquis is not a corpus}

JRC Acquis is:

\begin{itemize}
\item a dump of raw text
\item full of encoding errors\footnote{At least, for Polish and Czech. YMMV.}
\item not reliably sentence aligned: industry practice is to realign
\item not annotated
\item not maintained
\end{itemize}

{\footnotesize (For contrast, EuroParl is actively maintained, contains document origin annotation, 
speaker's original language annotation, and PoS annotation. Unfortunately, it contains
neither Czech nor Polish)}
\end{frame}
\begin{frame}{Why not SMT?}
\begin{description}
  \item[Hierarchical/Syntax Augmented] \hfill \\
  No available tree parsers.
  \item[Factored models] \hfill \\
  {\em Seems} promising
  \item[``Phrase'' Based] \hfill \\
  The only real option
\end{description}
\end{frame}

\begin{frame}{Why not SMT: Factored models}

\begin{quote}
One example to illustrate the short-comings of the
traditional surface word approach in statistical machine
translation is the poor handling of morphology.
Each word form is treated as a token in itself.

\ldots

While this problem does not show up as strongly
in English -- due to the very limited morphological
inflection in English -- it does constitute a significant
problem for morphologically rich languages
such as Arabic, German, Czech, etc.
\end{quote}

{\footnotesize
\textbf{Factored Translation Models}, {\em Philipp Koehn and Hieu Hoang}, EMNLP 2007, 
\url{http://homepages.inf.ed.ac.uk/pkoehn/publications/emnlp2007-factored.pdf}.}
\end{frame}

\begin{frame}{Why not SMT: Factored models}

The Official Version:

\begin{quote}
We reported on experiments that showed gains
over standard phrase-based models, both in terms
of automatic scores (gains of up to 2\% BLEU), as
well as a measure of grammatical coherence. These
experiments demonstrate that within the framework
of factored translation models additional information
can be successfully exploited to overcome some
short-comings of the currently dominant phrase-based
statistical approach.
\end{quote}

\hbox{}

{\footnotesize\em Ibid.}
\end{frame}

\begin{frame}{Why not SMT: Factored models}

The Unofficial Version:

\begin{quote}
``Factored models don't work.''
\end{quote}

\hbox{}

{\footnotesize 
-- \$WELL\_KNOWN\_SMT\_GUY

Unfortunately, BibTeX does not allow:

\begin{verbatim}
@PubConversation{wmt2010,
 author    = "Off the Record",
 title     = "Factored Models",
 year      =  2010,
 where     = "Lanigan's Plough",
}
\end{verbatim}
}

\end{frame}

\end{section}

\end{document}
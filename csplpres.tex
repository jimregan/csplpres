\documentclass{beamer}
\usepackage[utf8x]{inputenc}
\usepackage[T1]{fontenc}
\usetheme{Copenhagen}

\title[Czech to Polish]{Shallow-transfer rule-based machine translation from Czech to Polish}

\author[Ruth, O'Regan] % (optional, for multiple authors)
{Joanna~Ruth\inst{1} \and Jimmy~O'Regan\inst{2}}
\institute
{
  \inst{1}%
  Gda\'{n}sk University of Technology \\
  {\tt joannaruth1@gmail.com}
  \and
  \inst{2}%
  Eolaistriu Technologies \\
  {\tt joregan@gmail.com}
}

\date{}

\begin{document}

\begin{frame}
\titlepage
\end{frame}

%\begin{frame}{Warning!}
%
%This presentation is only loosely related to the paper.
%
%Think of this as more like the ``Special Features'' 
%section on a DVD.
%
%\end{frame}

\section{Introduction}
\begin{frame}{Czech and Polish}
\begin{quote}
In the 10th century, Czech and Polish were still basically the same language, 
which then began to diverge from each other, but even until the 14 century, 
Czechs and Poles understood each other without problems.
\end{quote}

\begin{center}
Czech Wikipedia, \emph{Polština}
\end{center}
\end{frame}

\begin{frame}{Czech and Polish: Similarities}
\begin{itemize}
\item Medium inflected:
  \pause
  \begin{itemize}
  \item 7 cases
  \item 3 genders
  \item Animacy distinction
  \end{itemize}
\pause
\item Relatively free word order:
  \pause
  \begin{itemize}
  \item Ala ma kota
  \item Kota Ala ma
  \item Ala kota ma
  \item Ma Ala kota
  \item \ldots
  \end{itemize}
\end{itemize}
\end{frame}
\end{section}

\begin{frame}{Czech and Polish: Cases}
\begin{center}
    \begin{tabular}{ | c | l | l |}
    \hline
    Case & Czech & Polish \\ \hline

    Nominative & matka & matka \\
    Genitive & matky & matki \\
    Dative & matce & matce \\
    Acusative & matku & matk\k{e} \\
    Instrumental & matkou & matk\k{a} \\
    Locative & matce & matce \\
    Vocative & matko & matko \\
    \hline
    \end{tabular}
\end{center}
\end{frame}

\begin{frame}{Czech vs. Polish: Viewpoints}

Polish view of Czech: Child-like

\begin{itemize}
\item More lexicalised diminutives.
\item Loss of palatalisation.
\end{itemize}

\emph{i.e.}, Czech sounds a little like Polish babytalk

Czech view of Polish: Archaic

\begin{itemize}
\item Digraphs (sz, cz) instead of caron.
\item Retention of Proto-Slavic ``nasal vowels''.
\end{itemize}
\end{frame}

\emph{i.e.}, written Polish looks like early written Czech.

\begin{frame}{An aside: ``l-participle''}

The Czech past form is sometimes referred to as the ``l-participle''.
Whether or not it's a participle is arguable.

\begin{itemize}
\item Not fully periphrastic: past.p3 uses no auxiliary. (Sorbian does)
\item Not fully adjectival.
\item Not a modifier.
\end{itemize}
\end{frame}

\end{section}

\end{document}